%%%%%%%%%%%%%%%%%
% This is an sample CV template created using altacv.cls
% (v1.1.4, 27 July 2018) written by LianTze Lim (liantze@gmail.com). Now compiles with pdfLaTeX, XeLaTeX and LuaLaTeX.
% 
%% It may be distributed and/or modified under the
%% conditions of the LaTeX Project Public License, either version 1.3
%% of this license or (at your option) any later version.
%% The latest version of this license is in
%%    http://www.latex-project.org/lppl.txt
%% and version 1.3 or later is part of all distributions of LaTeX
%% version 2003/12/01 or later.
%%%%%%%%%%%%%%%%
\newcommand{\RNum}[1]{\uppercase\expandafter{\romannumeral #1\relax}}
%% If you need to pass whatever options to xcolor
\PassOptionsToPackage{dvipsnames}{xcolor}

%% If you are using \orcid or academicons
%% icons, make sure you have the academicons 
%% option here, and compile with XeLaTeX
%% or LuaLaTeX.
% \documentclass[10pt,a4paper,academicons]{altacv}

%% Use the "normalphoto" option if you want a normal photo instead of cropped to a circle
% \documentclass[10pt,a4paper,normalphoto]{altacv}

\documentclass[10pt,a4paper]{altacv}
%% AltaCV uses the fontawesome and academicon fonts
%% and packages. 
%% See texdoc.net/pkg/fontawecome and http://texdoc.net/pkg/academicons for full list of symbols.
%% 
%% Compile with LuaLaTeX for best results. If you
%% want to use XeLaTeX, you may need to install
%% Academicons.ttf in your operating system's font 
%% folder.


% Change the page layout if you need to
\geometry{left=1cm,right=9cm,marginparwidth=6.8cm,marginparsep=1.2cm,top=1.25cm,bottom=1.25cm,footskip=2\baselineskip}

% Change the font if you want to.

% If using pdflatex:
\usepackage[T1]{fontenc}
\usepackage[utf8]{inputenc}
\usepackage[default]{lato}
\usepackage{tikz}
\newcommand{\ExternalLink}{%
    \tikz[x=1.2ex, y=1.2ex, baseline=-0.05ex]{% 
        \begin{scope}[x=1ex, y=1ex]
            \clip (-0.1,-0.1) 
                --++ (-0, 1.2) 
                --++ (0.6, 0) 
                --++ (0, -0.6) 
                --++ (0.6, 0) 
                --++ (0, -1);
            \path[draw, 
                line width = 0.5, 
                rounded corners=0.5] 
                (0,0) rectangle (1,1);
        \end{scope}
        \path[draw, line width = 0.5] (0.5, 0.5) 
            -- (1, 1);
        \path[draw, line width = 0.5] (0.6, 1) 
            -- (1, 1) -- (1, 0.6);
        }
    }

% If using xelatex or lualatex:
% \setmainfont{Lato}

% Change the colours if you want to
\definecolor{Navy}{HTML}{000080}
\definecolor{SlateGrey}{HTML}{2E2E2E}
\definecolor{LightGrey}{HTML}{444444}
\colorlet{heading}{Navy}
\colorlet{accent}{Navy}
\colorlet{emphasis}{SlateGrey}
\colorlet{body}{LightGrey}

% Change the bullets for itemize and rating marker
% for \cvskill if you want to
\renewcommand{\itemmarker}{{\small\textbullet}}
\renewcommand{\ratingmarker}{\faCircle}
%% sample.bib contains your publications
\addbibresource{sample.bib}

\usepackage[colorlinks]{hyperref}

\begin{document}
\definecolor{coolblack}{rgb}{0.0, 0.10, 0.14}
\name{Aaryan Raj Saxena}
\tagline{Computer Science Undergraduate, IIIT Delhi}
% \photo{2.8cm}{Globe_High}
\personalinfo{%
  % Not all of these are required!
  % You can add your own with \printinfo{symbol}{detail}
  \email{\color{black}{aaryan20004@iiitd.ac.in}}
  \phone{\color{black}{+91-9818481187}}
%  \mailaddress{Address, Street, 00000 County}
  \href{https://goo.gl/maps/TWmoBrRu9JgPqRYR8}{\location{\color{black}{New Delhi, India}}}
  \href{https://aaryan-r-s.github.io/Portfolio}{\homepage{\color{black}{Portfolio}}}
%  \twitter{@marissamayer}
  \href{https://www.linkedin.com/in/aaryan-raj-saxena-7016a1212}{\linkedin{\color{black}{LinkedIn}}}
  \href{https://github.com/Aaryan-R-S}{\github{\color{black}{GitHub}}}
  \color{coolblack}\boldsymbol{</>}\href{https://codeforces.com/profile/aaryan20004}{\color{black}{ Codeforces}}
  % I'm just making this up though.
  %% You MUST add the academicons option to \documentclass, then compile with LuaLaTeX or XeLaTeX, if you want to use \orcid or other academicons commands.
%   \orcid{orcid.org/0000-0000-0000-0000}
}





%% Make the header extend all the way to the right, if you want. 
\begin{fullwidth}
\makecvheader
\end{fullwidth}

%% Depending on your tastes, you may want to make fonts of itemize environments slightly smaller
% \AtBeginEnvironment{itemize}{\small}


%% Provide the file name containing the sidebar contents as an optional parameter to \cvsection.
%% You can always just use \marginpar{...} if you do
%% not need to align the top of the contents to any
%% \cvsection title in the "main" bar.
\cvsection[page1sidebar]{Experience}

\cvevent{\color{black}{Software Development Intern}}{Company Name \hspace{0.02cm}\href{https://www.microsoft.com/en-in}{\color{coolblack}{\ExternalLink}} }{May 202X -- June 202X}{}
\begin{itemize}
\item Lorem ipsum dolor sit amet, consectetur adipiscing elit, sed do eiusmod tempor incididunt ut labore et dolore magna aliqua. Ut enim ad minim veniam.
\end{itemize}

\divider

    \cvevent{\color{black}{Front End Development Intern}}{Company Name \hspace{0.02cm}\href{https://www.microsoft.com/en-in}{\color{coolblack}{\ExternalLink}} }{July 202X -- August 202X }{}
\begin{itemize}
\item Lorem ipsum dolor sit amet, consectetur adipiscing elit, sed do eiusmod tempor incididunt ut labore et dolore magna aliqua. Ut enim ad minim veniam.
\end{itemize}












\cvsection{Projects}


\cvevent{Web Portfolio}
{\MakeLowercase{HTML, CSS, jQuery, Bootstrap, AngularJS, Git}}{April 2021 -- May 2021}{}
\begin{itemize}
\item Developed a Web Portfolio for myself using AngularJS based on component-specific requirements for the app and published it to the GitHub pages as static webpage.
\item Links: \hspace{0.5cm}{{\href{https://github.com/Aaryan-R-S/Portfolio}{\github{\color{black}{GiHub}}}} {\href{https://aaryan-r-s.github.io/Portfolio}{\homepage{\color{black}{Link}}}}}
\end{itemize}
\divider

\cvevent{Hows That - Cricket Fantasy App}
{\MakeLowercase{html, Bootstrap, jquery, Python, requests, flask, gspread}}{November 2020 -- December 2020}{}
\begin{itemize}
\item Developed a fantasy Cricket team Site named How's That using python flask module, python gspread module and Cric-api.
\item Links: \hspace{0.5cm}{{\href{https://github.com/Aaryan-R-S/Web-Projects-Hows-That}{\github{\color{black}{GiHub}}}} {\href{https://hows-that-aaryanars.herokuapp.com/}{\homepage{\color{black}{Link}}}}}
\end{itemize}
\divider

\cvevent{Ping Pong Game}
{\MakeLowercase{HTML, canvas, CSS, JavaScript, Git}}{September 2020 -- October 2020}{}
\begin{itemize}
\item Developed a Ping Pong game using pure HTML Canvas and served as a static webpage using GitHub pages.
\item Links: \hspace{0.5cm}{{\href{https://github.com/Aaryan-R-S/Canvas-Projects-Ping-Pong}{\github{\color{black}{GiHub}}}} {\href{https://aaryan-r-s.github.io/Canvas-Projects-Ping-Pong}{\homepage{\color{black}{Link}}}}}
\end{itemize}
\divider



\cvevent{Postman Clone}
{\MakeLowercase{HTML, Bootstrap, javaScript fetch api, Git}}{October 2020 -- November 2020}{}
\begin{itemize}
\item Developed a Postman Clone Website using JavaScript Fetch API.
\item Links: \hspace{0.5cm}{{\href{https://github.com/Aaryan-R-S/Web-Projects-Postman-Clone}{\github{\color{black}{GiHub}}}} {\href{https://aaryan-r-s.github.io/Web-Projects-Postman-Clone}{\homepage{\color{black}{Link}}}}}
\end{itemize}
\divider




\begin{comment}

\cvevent{Native Speech Recognition}{}{}{}
\begin{itemize}
    \item Develop a Recommendation system which shows three kinds of recommendations based on the popularity of the genre,content
\end{itemize}
\medskip
\divider


\cvevent{ChatBot}{}{}{}
\begin{itemize}
\item Develop a Recommendation system which shows three kinds of recommendations based on the popularity of the genre,content
\end{itemize}
\medskip
\divider

\cvevent{Quiz Game}{}{}{}
\begin{itemize}
\item Develop a Recommendation system which shows three kinds of recommendations based on the popularity of the genre,content
\end{itemize}
\medskip
\divider

\cvevent{Whatsapp Broadcast}{}{}{}
\begin{itemize}
\item Develop a Recommendation system which shows three kinds of recommendations based on the popularity of the genre,content
\end{itemize}
\medskip
\divider
\end{comment}

% Adapted from @Jake's answer from http://tex.stackexchange.com/a/82729/226
% \wheelchart{outer radius}{inner radius}{
% comma-separated list of value/text width/color/detail}


\clearpage




%% If the NEXT page doesn't start with a \cvsection but you'd
%% still like to add a sidebar, then use this command on THIS
%% page to add it. The optional argument lets you pull up the 
%% sidebar a bit so that it looks aligned with the top of the
%% main column.
% \addnextpagesidebar[-1ex]{page3sidebar}

\end{document}
